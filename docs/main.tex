\documentclass[a4paper]{article} 
\input{head}
\begin{document}

%-------------------------------
%	TITLE SECTION
%-------------------------------

\fancyhead[C]{}
\hrule \medskip % Upper rule
\begin{minipage}{0.295\textwidth} 
\raggedright
\footnotesize
\textbf{Jaime Andres Torres Bermejo} \hfill\\   
202014866\hfill\\
andrestbermejoj@gmail.com\hfill\\

\textbf{Santiago Latorre Munar}\hfill\\
202111851\hfill\\
s.latorre@uniandes.edu.co

\textbf{Santiago Forero Gutiérrez} \hfill\\
202111446 \hfill\\
s.forerog2@uniandes.edu.co

\end{minipage}
\begin{minipage}{0.4\textwidth} 
\centering 
\large 
Caso 2\\ 
\normalsize 
Infraestructura Computacional\\ 
\end{minipage}
\begin{minipage}{0.295\textwidth} 
\raggedleft
\today\hfill\\
\end{minipage}
\medskip\hrule 
\bigskip

%-------------------------------
%	CONTENIDO
%-------------------------------
\section{Detalles de implementación}
\subsection{Modo 1}
Para lograr generar referencias de página en primer lugar se define una
posición inicial en páginas y desplazamiento para cada una de las 3 
matrices dependiendo del tamaño del entero, el tamaño de una página y 
el tamaño de la matriz. Una vez guardadas en distintas variables la 
página y desplazamiento inicial de cada matriz se inicia un ciclo para 
calcular la posición de cada entero de cada matriz. Para lograr esto 
se itera con un doble ciclo cada posición de las matrices (pues 
tienen mismo número de filas y columnas de modo que las posiciones 
coinciden). Para cada posición y para cada una de las 3 matrices se 
guarda en un String (que posteriormente se transformara en un archivo) 
la página y desplazamiento actual del entero de la posición actual de 
cada matriz. Es decir que por cada ciclo se almacena la posición 
equivalente de las tres matrices (pues así se calcula la suma). 
Posteriormente se actualiza la página y desplazamiento actual de cada 
matriz usando divisiones enteras y residuos. Para actualizar la página 
se le suma a la variable “el valor de sumar el desplazamiento actual y 
el tamaño de entero, dividido en una división entera por el tamaño de 
una página”. De forma que si la suma del desplazamiento actual y el 
tamaño de entero es menor al tamaño de página, la página permanece 
igual; y si es mayor, se aumenta en una posición la página. Respecto al 
desplazamiento se realiza la misma operación, pero en lugar de sumar a 
la variable se realiza una asignación y en lugar de hacer división 
entera se realiza la operación con módulo. De forma que si la suma del 
desplazamiento actual y el tamaño de entero es menor al tamaño de 
página, el desplazamiento se le suma el tamaño de entero; y si es 
mayor, se vuelve a iniciar en 0. Finalmente se trasforma esto a un 
archivo.

\subsection{Modo 2}
\subsubsection{Algoritmo de Envejecimiento.}
Con respecto al algoritmo de envejecimiento, este se encuentra localizado
en el Manejador, principalmente con el fin de poder acceder a la tabla de 
páginas de forma relativamente sencilla, cambiando valores como la edad binaria
de la página o el booleano que indica en memoria principal si se encuentra
o no. Esto reduce el tiempo de ejecución de la operación, lo cuál es importante
cuando consideramos que esta es una operación realizada varias veces por segundo.

Las condiciones y parámetros de las páginas estan definidos en la clase 'Página',
por lo que los 2 algoritmos principales del proceso de envejecimiento pueden
consultar una misma estructura y hacer cambios acordemente. El primer algoritmo
itera sobre la tabla de páginas, haciendo operaciones binarias sobre el int que
representa la edad de la página, mientras que el otro selecciona la página de
menor edad.

\section{Gráficos de ejecución}
\subsection{Prueba 1}
\includegraphics{1-1.jpeg}
\includegraphics[scale=0.55]{1-2.jpeg}

\subsection{Prueba 2}
\includegraphics{2-1.jpeg}
\includegraphics[scale=0.55]{2-2.jpeg}


%-------------------------------
\section{Preguntas realizadas.}
\subsection{¿Qué estructuras de datos se utilizaron y porque? (cuándo se actualizan, con base en qué y en qué consiste la actualización). }
En general, esta solución al problema no requiere ninguna dependencia que
no esté en la base de Java 1.8 Themurin y, al leer esta solución, es
bastante evidente que fue en realidad una solución bastante orientada a
trabajar con las fortalezas de Java. Entonces pues, encontremos las clases
propuestas y con ellas, sus atributos. Si fuesemos a considerar una clase
como una estructura de datos de alto nivel, bajo la lógica del lenguaje usado,
nos daríamos cuenta de que la clase 'Pagina.java' es en realidad, bastante
parecida a como se utilizaría un struct en C, y que su principal función es
almacenar y actualizar información de una página en concreto. Esta luego se
almacenará en arrays de Páginas, que representarán las estructuras del sistema
operativo que almacenan estas páginas de memoria. Estos arrays corresponden a
MemoriaVirtual, MarcoPagina y TablaDePaginas, cuyos nombres son indicativos de
las definiciones a estos términos establecidas de forma teórica, con MemoriaVirtual
apuntando a todas las tablas contenidas en la memoria virtual, MarcoPagina indicando
las páginas en memoria real, y TablaDePaginas siendo un array complementario a
este último, que almacena la información de indices dentro de la memoria real. 
Iterando sobre estos arrays podemos consultar fácilmente las páginas dentro de ciclos y acceder a estas
durante la ejecución de los algoritmos. Esto es particularmente evidente en
el modo 2, donde el algoritmo de envejecimiento y de actualización constantemente
deben acceder y modificar páginas, en caso de ser de interés para un algoritmo
en concreto, este accederá a la información contenida y la cambiará, como los
algoritmos estan sincronizados podemos garantizar hasta cierto punto la 
consistencia interna de la información y poder mantener registro de todo,
fuera de esto se llevó cuenta de valores númericos importantes con int y se
usaron Scanners para poder leer y escribir archivos en el programa. El manejo de
estas estructuras puede en algunos casos llegar a tener en cuenta el orden de las
operaciones, como ocurre con MarcoPagina, por ejemplo.

\subsection{¿Que esquema de sincronización fue usado? Justifique brevemente dónde es necesario usar sincronización y por qué}

Para lograr la sincronización entre los threads del algoritmo de reemplazo de páginas y el de envejecimiento se usa una clase monitor que se instancia una única vez para almacenar las estructuras compartidas entre ambos threads. Dentro de esta instancia única de monitor se tienen métodos sincronizados (synchronized) sobre los get de varias variables o sobre la variable “continuar” que permanecerá en verdadero hasta que termine el thread de reemplazo de páginas para avisar al de envejecimiento que debe terminar. O también sobre el método de almacenar memoria virtual donde se crean las páginas con los valores por defecto, esto ya que si los otros hilos comienzan a ejecutarse antes de instanciar todas las páginas podrían haber lecturas sucias del tamaño de la memoria virtual. También sobre el método agregar página ya que se encarga de forma atómica de reemplazar páginas en marcos de página cuando haya un fallo de página (eliminación y agregación), además de que consulta el resultado del algoritmo de envejecimiento. Y, finalmente, sobre el método de envejecimiento ya que este no puede correr a la vez que el otro thread o si no podría haber una página usada recientemente que no sea tomada en cuenta por el algoritmo de envejecimiento.

\subsection{Escriba su interpretación de los resultados: ¿tienen sentido? Justifique su respuesta.}

Como se puede evidenciar en los dos conjuntos de datos, existe una clara relación entre el número de fallas, el tamaño de página y el tamaño de la matriz, particularmente con el tamaño de la página, donde al disminuir este valor aumenta la cantidad de fallas. Al revisar la teoría esto cobra sentido, ya que para almacenar la misma información se requiere recurrir a una mayor cantidad de páginas, aumentando también la cantidad de páginas que se deben cargar en memoria real y por ende generando fallas con una mayor frecuencia. En la gráfica \#1, se puede observar un pico especialmente alto con un tamaño de pagina de 256 y una matriz 64x64, siendo este un claro ejemplo de lo que se menciona, ya que es una gran cantidad de valores que se guardan en páginas de tamaño reducido, generando una gran cantidad de paginas y, por ende, una gran cantidad de fallos. Por otro lado, al comparar las dos gráficas entre sí, se puede observar que disminuir la cantidad de marcos también resultará en un incremento muy notable de las fallas, debido a que se tienen que estar cambiando constantemente las páginas que se encuentran en memoria real. 

Observando más a fondo la grafica \#2, se puede observar que a medida que aumenta el tamaño de página, la cantidad de fallos disminuye. Sin embargo este efecto se ve opacado al incrementar mucho el tamaño de la matriz, pues son tantos los valores que se guardan en memoria virtual, que sin importar el tamaño de la página, siempre se generará una cantidad de paginas exponencialmente mayor a la de marcos de página, dando como resultado una tasa de fallos relativamente constante en las matrices más grandes, sin importar el tamaño de página que se maneje.


\subsection{¿Que pasaría si las matrices fueran almacenadas en major column order?}
En caso de que pudiésemos cambiar el orden de ejecución de la suma no habría mayor diferencia, pues sería muy similar al caso actual. Sin embargo, si la suma se siguiera haciendo en orden de las filas pero las matrices fueran almacenadas siguiendo column-major order ya se empezaría a notar la diferencia. Ya que analizando como funciona el algoritmo actualmente, podemos ver que cuando en los marcos de página caben al menos tres matrices a la vez, no se generan tantos fallos al leer un gran número de elementos de una misma página en una sola subida a memoria física. Es decir que es probable que los elementos ubicados en una misma fila estén en la misma página y no haya tantos fallos de página. Mientras que en un column-major order la probabilidad de que dos elementos de una misma fila se encuentren en una misma página disminuiría bastante. Por lo que es posible que para cada elemento de la matriz no se encuentre la página necesaria en los marcos de página, aumentando considerablemente los fallos de página.

\begin{center}
\includegraphics[scale=0.25]{cmo.jpeg}
\end{center}

\subsection{¿Cómo varía el número de fallas de página si el algoritmo 
estudiado es el de multiplicación de matrices?}
La suma es una operación cuyo resultado generalmente no excede el 
tamaño de los sumandos, es decir, no requiere de una mayor cantidad de 
bits para ser representado en su forma binaria. Teniendo esto en cuenta 
y considerando que en el algoritmo solo se estudiaron tamaños de entero 
inferiores al tamaño de una página, se puede afirmar que para cargar los 
tres números de la suma (cada uno proveniente de su respectiva matriz) 
se requiere de máximo tres marcos de página. Por otro lado, dada la 
naturaleza de la multiplicación, es muchísimo más probable que el 
resultado adquiera un valor superior al de los factores, de manera que, 
en las mismas condiciones de la suma, será necesaria más de una página 
para almacenar este valor en su totalidad. Esto ya que para multiplicar 
matrices se tendría que realizar un producto punto para cada posición 
de la matriz resultado (recordar la fórmula de producto punto):

\begin{center}
    \includegraphics[scale=0.5]{prod_p.jpeg}    
\end{center}


Y, además, este calculo se realiza sobre la fila de una matriz y la columna de la otra, de forma que sería probable que se deban usar más páginas para guardar los cálculos parciales. Habiendo comprendido como se desenvuelven ambos algoritmos al cargarse en memoria virtual, llega el momento de revisar qué sucede cuando se quieren cargar a memoria real, lugar donde suceden las fallas de página: como ya se ha comprobado que en promedio una operación de multiplicación requiere de más páginas que una de suma, también necesitará de más marcos de página para cargarse en memoria real. Como resultado, se producirá una mayor cantidad de fallas al ser necesario reemplazar más frecuentemente las páginas para poder cargar los números que se requieran.



\end{document}
